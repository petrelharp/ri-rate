\documentclass{article}
\usepackage{amsmath,amssymb}
\newcommand{\E}{\mathbb{E}}
\renewcommand{\P}{\mathbb{P}}
\begin{document}

Question: Should the inferred rate of accumulation of reproductive isolation covary positively with the rate of speciation?


\paragraph{Simplest model:} Rate of speciation in allopatry is $s$;
allopatric populations split off, and stay allopatric for time $T$, at rate $a(T)$.
Rate of speciation is then
\begin{align}
  \int_0^\infty a(T) (1-e^{-sT}) dT .
\end{align}
Two simple cases: if it is sufficiently rare that $T$ is really large,
and $sT$ is small, then this is approximately $s$ times the rate of population splits times the mean time in allopatry.
If it's the rare, large $T$ that matter, then this will just be the mean rate of appearance of such large $T$.


\paragraph{Non-binary model:} DMIs accumulate at constant rate $\lambda$;
RI is multiplicative in the number of DMIs.


\paragraph{Varying $s$:}
And, what about when $s$ varies along the lineage?
One calculation: suppose $s$ evolves as an Ornstein-Uhlenbeck process,
and the instantaneous speciation rate is as given above.
Run this for an amount of time so that you'd expect there to have been one or two speciation events.
How strong is the correlation between the mean $s$ across the observed lineages and the number of species observed?
There would probably be a somewhat sharp transition between when it is well correlated and when it isn't hardly at all, 
as one increases the rate at which $s$ evolves.

This process would be a continuous-type branching process.

Question: what could we calculate from this model that is closer to inferring $s$ ancestrally on the tree?

\paragraph{Simpler calculations:}
If most $T$ are small enough that speciation is unlikely,
but then there are really long $T$ that occur simultaneous in all species;
and if these $T$ are so long that speciation is certain,
there should be no correlation.

This could be convincing with the right order of magnitude of calculation of $s$ and $T$?


%%%%
\section{Math, first simple model}

Suppose that RI occurs instantly, and that the probability that RI has not occurred before time $\tau$ is equal to $\exp(-r(\tau))$.
(We'll mostly take $r(\tau) = r\tau$, i.e.\ speciation happens instantly, after an exponentially distributed time with rate $r$.)
Suppose that the expected number of populations that split off between $t$ and $t+dt$,
that are destined to remain allopatric until $t+\tau$,
is $a(\tau) dt$; and that these points form a Poisson process.
The mean rate of initiation of population splits that are destined to speciate is
the integral over possible durations of allopatry
of the rate of such splits multiplied by the probability that RI evolves before secondary contact, or
\[
 \gamma := \int_0^\infty (1-e^{-r(\tau)}) a(\tau) d\tau  .
\]

If $r(\tau) = r \tau$ (single-trait speciation)
and $a(\tau) = a_0 \lambda e^{-\lambda \tau}$ (rate $a_0$ appearance of splits that last for an exponential time with mean $1/\lambda$)
then
\[
\gamma = a_0 \int_0^\infty (1-e^{-r\tau}) \lambda e^{-\lambda \tau} d\tau = \frac{a_0 r}{r+\lambda} .
\]
This is the rate of appearance of new species under.

However, note that at any particular time there may exist currently allopatric populations that are destined to speciate,
but secondary contact has not yet occurred.
The mean number of such (for which RI may or may not have already occurred)
is the integral over pairs $(t,\tau)$ of (amount of time ago the split occurred, duration of the split, with $t < \tau$)
of the chance RI occurs:
\[
\int_0^\infty \int_t^\infty (1-\exp(-r(\tau))) a(\tau) d\tau dt .
\]
In some of these, RI has not yet occurred, but will;
the mean number of such is
\[
\int_0^\infty e^{-r(t)} ( 1- e^{-r(\tau-t)} ) a(\tau) d\tau dt .
\]
\emph{Note:} Is this model well-defined?  
What happens if: $A$ splits off $B$, 
which evolves RI at time $s$ before secondary contact at time $t>s$ with $A$,
but $B$ also splits off $C$ before evolving RI, at time $u<s$; and perhaps also $C$ evolves RI at time $v<s$.


If the driving process that throws off allopatry events is Poisson,
then speciation events are equally likely to occur on any lineage,
and so the number of to-be-species
is just a Yule process of rate $\gamma$;
but with the caveat that we might not get to see recent splits.
If we evolve starting with a single species for time $T$, 
let $N_T$ be the number of splits that will eventually lead to species,
and $S_T$ the number of realized species already,
then $N_T$ is Yule with rate $\gamma$, 
so has $\E[N_T] = e^{\gamma T}$,
and $S_T \le N_T$ is more complicated.
But: since the probability of having a split that evolves RI before the present
only depends on how long ago the split was,
the process $(S_T(t))_{t=0}^T$ that records how many distinct ancestors of today's species there were at time $t$
is a time-inhomogeneous Yule process,
with instantaneous split rate at time $T-u$ equal to
\[
\gamma_T(u) = \int_0^\infty (1-e^{-r(\min(\tau,u))}) a(\tau) d\tau .
\]
Therefore, the total ``Yule-time'', for which $\E[S_T] = e^{s(T)}$, is 
\begin{align}
  s(T) &:= \int_0^T \gamma_T(u) du \\
  &= \int_0^T \left\{ \int_0^t (1-e^{r(\tau)}) + \int_t^\infty (1-e^{-r(t)}) \right\} a(\tau) d\tau dt ,
\end{align}
and if $a(\tau) = a_0 \lambda e^{-\lambda \tau}$ and $r(t) = rt$ then
\begin{align}
  s(T) &= a_0 \int_0^T \left\{ \int_0^t (1-e^{-r\tau}) + \int_t^\infty(1-e^{-rt}) \right\} \lambda e^{-\lambda \tau} d\tau dt \\
  &= a_0 \int_0^T \left\{ \int_0^\infty \lambda e^{-\lambda \tau} d\tau - \int_0^t \lambda e^{-(r+\lambda) \tau} d\tau - \int_t^\infty \lambda e^{-(rt+\lambda\tau)} d\tau \right\} dt \\
  &= a_0 \int_0^T \left\{ 1- \frac{\lambda}{r+\lambda} (1-e^{-(r+\lambda) t}) - e^{-(r+\lambda)t} \right\} dt \\
  &= a_0 \left\{ T \left( 1 - \frac{\lambda}{r+\lambda} \right) + \frac{\lambda}{(r+\lambda)^2} (1-e^{-(r+\lambda) T}) - \frac{1}{r+\lambda} \left( 1 - e^{-(r+\lambda)T} \right) \right\} \\
  &= a_0 \left\{ \frac{rT}{r+\lambda} + \frac{r}{(r+\lambda)^2} (1-e^{-(r+\lambda) T}) \right\} \\
  &= \frac{a_0 r}{r+\lambda} \left( T + \frac{ 1-e^{-(r+\lambda)T} }{ r+\lambda } \right) .
\end{align}
Note this is more or less linear in $T$; we want to know what relation it has to $r$,
across the relevant range of $r$.

%%%
\section{A wilder model}

Thoughts on a more realistic model:
Really, rate of splits of allopatric populations and accumulation of DMIs should depend on things like range size and population size.
If a population splits in half, well, the two halves should be half as big as the original.
A more realisticish model would go something like this:

Without explicitly modeling geography, you could imagine a fragmentation-coalescent process, 
where each piece becomes a new species at some rate that depends on how big it is.
Or, you could suppose that each piece accumulates mutations at some rate that depends on its size,
each such mutation is unique,
and when two pieces coalesce, some of the mutations are lost, so the set of mutations carried by the combined piece
is a subset of the mutations carried by the two smaller pieces.
If the rate of coalescence depends on the number of segregating mutations,
then we'd have a model of speciation.

To explicitly model geography, you could say that the state of the proces at any time
is a partition of the total range into sub-ranges,
and that each sub-range has its own set of mutations,
and there is some driving Poisson process of vicariance events that add and remove boundaries between sub-ranges.
Only adjacent pieces can coalesce; otherwise the model would be as above.

Or, we could take this model of geography,
but assume that speciation is instant at the first mutation;
this seems do-able.

\paragraph{Questions/notes:}
\begin{itemize}
  \item How should the rate of accumulation of RI depend on the population size?
  \item How should the total range size affect the rate of allopatry?
\end{itemize}


\end{document}
